% Author: Eskil Joergensen
% Tex Author: Magdalen Berns
% Formatting: Magdalen Berns

\chapter{Future Research of Retinal Imaging and Analysis}

\label{future_research}
\lhead{\emph{Future Research in Retinal Imaging}}
\section{Future Research in Retinal Imaging}
\label{future}

\section{Cheap Retinal Scanners for global accessibility}

The development of an economic and accurate system for an early
glaucoma and diabetic retinopathy. As the most common roots of
blindness, an early detection of both conditions are vital for
curing the diseases. The damages done at the point of discovery
are non-reversible. The development of an easy-to-use cost-effective
apparatus would mean patients can be checked and hence be treated
more easily.

With rapidly growing number of patients due to lifestyle-illnesses,
including a diabetes (type II), health institutions would be able
to save both capacity and resources if such tools were readily
accessible. An early diagnosis of diabetic retinopathy reduces
the risk of blindness of 50\%.\cite{}

Today, besides the equipment, the study of retinal images in medicine
is expensive and time consuming as an ophthalmologist has to analyse
the images in order to make a diagnosis. This is a process that can
be automated by computers.  In 1996 a correct identification rate of
88.9\%
was achieved by automated computer-analysis, which means that the
retinal images can be taken and analysed automatically. The potential
patients can then be filtered and referred to a health professional
to receive proper treatment.\cite{}

( With an increasing number of connections associating the health
of the retina with modern lifestyle-illnesses, being able to produce
cheap and reliable retinal imagery apparatus is an important step for
early identification and swift treatment of such deceases. ) With
recent advances in the state of wearable technology, personal devices
can now monitoring heart rates, sleeping cycles and hydration levels.
Having personal devices that are able to monitor the stress (and
general health condition) of the eye would be the idea solution.
Currently there are composite devices that are able to take "simple"
images of the retina\cite{}.
 
\section{Blue light hazard}

The effect of blue light LED on macular degeneration and retinal
damage. With fast-growing use of LED devices and energy-efficient
light bulbs, with a peak light-wavelength in the blue light region,
can cause increased oxidative stress on the retinal tissue.\cite{}
A chronic exposure to blue light can result in induced photochemical
deterioration of the retina, which has the effect of accelerating the
biological ageing. [Behar-Cohen F, Martinsons C, Vienot F, Zissis G,
BarlierSalsi A, Cesarini JP, et al. 2011. Light-emitting diodes (LED)
for domestic lighting: any risks for the eye?\cite{}

As everyday use of blue-intensive light sources is relatively new,
the long-term effects can be serious macular degeneration. Hence the
development of retinal imaging techniques are important in maintaining
the study of these trends and also better understanding the biological
effects on the retina.
 
Normal white light contains a wide distribution of "all" colour
wavelengths. Waves of different wavelength are either absorbed or
transmitted by the different sections of the eye. As a result,
being exposed to natural sunlight means that the different sections
of the eye are all used. By introducing a narrow range of light waves
means that some parts of the eye will be used more than others. In the
case of blue-intensive light waves, those are all absorbed by the retina
alone, as can be seen in the table [Light-emitting diodes (LED) for
domestic lighting: Any risks for the eye?\cite{}

\section{Multimodality}

Multimodality imaging is becoming increasingly common in
ophthalmology. For image information from multiple modalities
to be usable in mutual context, images must be registered so
that the independent information that was acquired by different
methods can be concatenated and form a multimodality description
vector. Thus, because of its importance in enabling multimodal
analysis, retinal image registration reflects another active
area of research. The several clinically used methods to image
the retina were introduced above and include fundus photography,
scanning laser ophthalmoscopy, fluorescence imaging, and OCT.
Additional retinal imaging techniques such as hyper-spectral imaging,
oxymetry, and adaptive optics SLO will bring higher resolution.

To achieve a comprehensive description of retinal morphology
and eventually function, diverse retinal images acquired by different
or the same modalities at different time instants must be mutually
registered to spatially combine all available local information.
The following sections provide a brief overview of fundus photography
and OCT registration approaches in both two and three dimensions.
Registration of retinal images from other existing and future imaging
devices can be performed in a similar or generally identical manner.

\section{Security}

Using the unique biometrics of the eye (iris and retina) one can set up
certain system to recognise these patterns. For security purposes the use
of the unique retina pattern has many advantages and applications. The
pattern of the retina, which maps the locations of veins, arteries the
macula and the optic nerve in the eye, has a much more complex pattern
than the normally used fingerprint or iris scan.\cite{} Personal
iris patterns and fingerprints are accessible to others, this means
they are more easily copied. Where a thumb can be amputated and still
be used in a scanner, the blood vessels mapping across the retina would
contract so quickly that having someones "amputated" eye would not help
in tricking a retina recognition scanner. As a result using retinal
imaging as the ultimate personal identification is possible.

In terms of consistency a retina scan has an error rate of 1 in 10 000 000,
whereas the error of a fingerprint scan can be as high as 1 in 500.\cite{}
