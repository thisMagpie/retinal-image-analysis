\chapter{Conclusion}

\label{conclusion}
\lhead{\emph{Conclusion}}

This review has sought to provide an extensive investigation
into the technology that underpins retinal imaging, from the
early origins of ophthalmoscopy in the 19th century, through the 
currently used fundus cameras, \gls{cslo} and \Gls{oct} machines and up 
to the most recent advancements in imaging technology utilizing 
adaptive optics and two-photon microscopy. Before this was done, 
a detailed look into the eye and its dysfunction, pathology, and 
injuries was carried out to provide the necessary background for 
retinal imaging research.

Initial retinal imaging devices were able to document a wide array 
of common pathology; with glaucoma, diabetic retinopathy and macular 
edema all observable using primitive fundus photography and 
ophthalmoscopy. The ability of the early machines to detect 
disease once it had manifested to a sight threatening level provided 
useful information to ophthalmologists, however, it was of little help 
in preventing vision loss. 

With improvements in image resolution in fundus cameras came 
the ability of devices to detect symptoms of disease development, 
resulting in the onset of early detection screening programmes. 
Following this, practitioners became concerned with imaging 
wider visual fields and achieving greater depth penetration into 
the retinal tissue, to learn more about the how certain diseases 
manifest themselves across the retina. This resulted in the 
development of the \gls{cslo} and the \Gls{oct}, which widened the scope 
of the visible retina by accessing the peripheral retina and obtaining 
higher resolution within retinal layers for closer study. Yet again 
expanding the diagnostic capability of eye-care professionals. 

These devices have been key in the progression of retinal disease 
study and have encouraged further technological developments. 
These developments aim to widen the scope of current machines 
and bring them to areas of the world that do not have adequate 
infrastructure to support widespread use. With new equipment and 
new diagnostic systems, retinal imaging technology will be able to 
provide a valuable clinical tool to help patients preserve their vision 
in the developing world by making them affordable and easy to use.
 
Increasing rates of diabetes and a globally ageing population are 
challenges that have to be met with the development of innovative 
methods. Specialists therefore must continue to develop high 
quality imaging devices to respond to global healthcare demands. 

