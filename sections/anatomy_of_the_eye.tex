% Author: Magdalen Berns

\chapter{The Physiology of The Eye}

\label{anatomy}
\lhead{\emph{The Physiology of The Eye}}
\section{A Normal, Healthy Eye}

The human eye is a complex biological system, which allows the brain to
form an image of its surroundings through the interpretation of
electromagnetic signals. \Fref{fig:eye_simple} shows a simple schematic
diagram of the layout of the eye. \todo[inline]{write a bit about about the direction of light}

\begin{figure}[!htbp]
 \centering
   \includegraphics{figures/schematic_diagram_of_the_human_eye}
 \caption{A simple schematic diagram of the layout of the eye.\cite{wikiRhcastilhos}}
 \label{fig:eye_simple}
\end{figure}

The cornea of the eye is a transparent layer around 0.6mm thick
which curves around the iris as well as the anterior
chamber.\cite{yaylali1997corneal,thoft1983x,patel1994refractive}
With a mean refractive index of around 1.4 (about the same as water),
the cornea allows plenty of light to pass through. Incident light refracts as in a
convergent lens due to the convex shape of the cornea.  \Eref{eq:refractive} shows
Snell's Law of refraction for a light wave passing through two
different isotropic materials which have refractive indices $n_1$
and $n_2$, respectively.

\begin{figure}[htbp]
 \centering
   \includegraphics{figures/snells}
 \caption{Snell's law for diffraction at an interface where $n_1$ \textless $n_2$.\cite{wikisnell}}
 \label{fig:snell}
\end{figure}

The angle $\theta_1$ is between the normal to the interface($n_1$ and $n_2$)
and the incident light ray and the angle $\theta_2$ is between the normal to
the interface($n_2$ and $n_1$) and the transmitted light ray, as shown in \fref{fig:snell}.

\begin{equation}
n_1\sin\theta_1=n_2\sin\theta_2
\label{eq:refractive}
\end{equation}

The outermost surface of the cornea is made of epithelial cells which
are cyclically lost and replenished.
\cite{jester1999cellular,hassell2010molecular} The reproduction
of cells is facilitated in part by tear ducts, which serve to
moisten the eyes and remove harmful bacteria.\cite{holly1977tear}

There is a small circular opening in the iris (the coloured section of
the eye) called the pupil, which has an aperture of around 7mm that
dilates to allow an appropriate amount of light to diffract through
the lens onto the retina.\cite{krugman1964some} Following the initial
refraction through the cornea, light is again refracted through the lens
to finely focus the image to a central point on the retina.

\begin{equation}
\frac{1}{S_1} + \frac{1}{S_2} = \frac{1}{f}
\label{eq:lens_makers}
\end{equation}

The lens maker’s formula in \eref{eq:lens_makers} is an expression which relates the
focal point of a lens to the distance of the object and image from the lens centre,
as is illustrated in \fref{fig:convergent_lens}.

\begin{figure}[htbp]
 \centering
   \includegraphics{figures/convergent_lens2}
 \caption{The convex lens focuses the object(distance $S_1$ from the lens centre) and inverts this image(at a distance $S_2$ from the lens centre). \cite{greivenkamp2004field}}
 \label{fig:convergent_lens}
\end{figure}


The angular resolution, $\theta$ of the focal point on the retina is limited by
the diameter $D$ of the pupil and is directly proportional to the wavelength of
the diffracted light, $\lambda$ as expressed in \eref{eq:res_limit}

\begin{equation}
\theta=\frac{1.22\lambda}{D}
\label{eq:res_limit}
\end{equation}

The lens is accommodated by a ciliary body of tissue which made up of fiber and muscle.
The ciliary body secretes a fluid, (known as aqueous humour) into a canal that flows
around the circumference of the eye, scleral venous sinus.
\cite{bill1970effects,dvorak1934schlemm} The primary function of the aqueous humour
is to maintain intraocular pressure by transferring fluids around the eye.

When the eye focuses on objects that are nearby, the ciliary body muscles contract,
causing the lens to flatten as a result of contractile forces. If an object is far
away the light rays can be approximated as parallel to the principal axis of the lens.
To focus on distant objects the suspensory ligaments relax to allow the lens to return
to a normal curved shape, hence light is refracted and focused onto the retina.

\begin{figure}[!htbp]
 \centering
   \includegraphics{figures/eye_diagram}
 \caption{A diagram of the eye showing of the visual
  and optic axis, the cornea, the retina and the fovea}
 \label{fig:optic_axis}
\end{figure}

Photons of light are refracted out of the lens before they pass through
a clear substance called the vitreous chamber and land onto the retina,
which is also transparent. A diagram indicating the optic axis and
visual axis is shown in \fref{fig:optic_axis}.

The retina is a membrane which covers the entire receptive field of
vision, it is part of the central nervous system.\cite{rogers1983neurite}
Just behind the retina are ganglion cells, biploar cells, cones and rods.
These are supported by pigment epithelium cells and the choroid - a vascular
bed of tissues which supply the retina with blood and removes toxins.
\cite{lutty1996localization} A schematic overview of the core retinal
constitution is given in \fref{fig:retina}

\begin{figure}[!htbp]
 \centering
   \includegraphics{figures/rods_and_cones}
 \caption{A schematic diagram of the retina with the direction of light indicated
 as being from the bottom upwards direction.}
 \label{fig:retina}
\end{figure}

There are around 0.7 to 1.5 million ganglion cells in a normal human retina.
\cite{curcio1990topography}. Retinal ganglion cells are a form of neurons and
are integral in the transmission of electrical signals to the brain.
\cite{meyer1995characterization} Cones and rods are photoreceptors that
convert light into electrical signals which are transmitted to the
brain from the optic nerve, via bunches of ganglion nerves.

Whilst cones are not particularly sensitive to light, they do aid to visual
acuity by granting us colour vision.\cite{bowmaker1980visual} Conversely, rods
which have a region of pigments around $1\mu{m}$, are particularly sensitive,
even to the pressure exerted by a single photon of light. Rods tend to be
located around the periphery of the receptive field to maximise peripheral light
collection.\cite{liebman1964sensitive,baylor1979responses} Most of the retinal
cones are located around a circular trough in the receptive field, called the
fovea centralis which is located at the centre of the macula.\cite{hendrickson1994primate} \Fref{rod_cone_density} is a graph of the density of cones and rods against their
angular distance from the fovea, which is the most sensitive part of the eye,
so most cones are located directly behind it. At the optic disc there are no 
photoreceptors so this region is known as the "blind spot".

\begin{figure}[!htbp]
 \centering
   \includegraphics{rod_cone_density}
 \caption{Plot of density of cones and rods against their angular distance from the fovea.
 The cone density peak lies at the fovea, where there is an absence of rods.
 \cite{rod_cone_density}}
 \label{fig:rod_cone_density}
\end{figure}


\Fref{fig:cone} shows a schematic diagram of the cone cell.

\begin{figure}[!htbp]
 \centering
   \includegraphics{cone_cell}
 \caption{A schematic diagram of the cone cell showing the outer and inner segments, nucleus
 and the synaptic terminal.\cite{wikicone}}
 \label{fig:cone}
\end{figure}

The fovea has an apparent dip, because unlike the periphery,
there are no neurons situated behind it. Neurons are cells which process
and transmit electrical information. The cell body is connected to the optic
nerve through an extended nerve fibre, known as an axon, which sends the visual
information directly to the brain. As the photoreceptors (mainly cones) beneath
the fovea are responsible for the acuity and colour of the centre field of vision,
axons would disturb the tissue and result in a distorted visual field.

A healthy and normal eye will pick up red, green and blue so that the brain
can interpret the full spectrum of colours. \Fref{fig:wavelengths}
shows normalised absorbance against wavelength for red, green and blue cones
and rods (which cannot differentiate between colours), in an average, human eye.
The range of human visual spectrum of light lies roughly between 390nm
and 700nm.\cite{starr2010biology}.

\begin{figure}[!htbp]
 \centering
   \includegraphics{wavelengths}
 \caption{Normalised absorbance vs. wavelength, for cones of an average eye.\cite{wikicones}}
 \label{fig:wavelengths}
\end{figure}

\section{Dysfunction}

As people age, ciliary muscles accommodating the lens weaken and
this impairs their ability to focus on objects which are close by, this
is known as presbyopia.\cite{fisher1985ciliary}


A common defect affecting vision is often referred to as "colour blindness".
This is the limited functionality of certain cones which are are sensitive
to specific frequencies of light but not others such that a person can fail
to differentiate certain colours of the spectrum. This defect is related to
the X chromosome and hence it seems to only affect some men, who if can be
born with defected cones.\cite{george1996clinical} If all the cones are defect,
then this would result in blindness.

\section{Pathology}

Age related Macular degeneration of pigment cells and decreases in
Equatorial rods and ganglion cell rates are common problems affecting
central vision.\cite{gao1992aging} Macular degeneration in particular,
accounts for 95\% of blind and partial sightedness registrations in the
UK and particularly affects women.\cite{o1998age,klein2005complement}
There are two main forms of macular degeneration, non-exudative (dry)
and exudative (wet). The phenotypes in on-exudative macular degeneration
are atrophic and neovascular respectively.\cite{kuno2011dry} In
non-exudative macular degeneration, deposits build up behind the retina
resulting in its progressive thinning or scarring, with respect to time.
Exudative macular degeneration is caused by leaking of blood vessels
which causes swelling.

Preliminary studies indicate that the thickness of the choroid decreases
with age and although little is understood about how this may affect
central vision in macular degeneration, some studies have found
choroid thickness to be a predictor for open angle glaucoma which
is yet another leading cause of blindness.
\cite{margolis2009pilot,gordon2002ocular}

Glaucoma is caused by various malfunctions of aqueous humour drainage
which can lead to excessive intraocular pressure bringing about damage
to the optic nerve and other vital members of the optical system.
\cite{distelhorst2003open} Open angle glaucoma happens gradually,
affecting the peripheral vision at first and progresses on to impair
central vision over time. For early diagnosis of open angle glaucoma
the relationship between intraocular pressure and visual field decay
is examined by specialists to determine whether any abnormalities are
apparent so that it may be treated before permanent field damage occurs.
\cite{goldmann1972open}

Retinopathy is another common disease but one which can affect people of
all ages, by attacking the retina. A common indicator of non-proliferative
retinopathy is macular edema, symptomatic blurred vision arising from
abnormal fluid leakage and swelling in the macular from the retina's
blood vessels causing the the retina to thicken.\cite{hee1995quantitative}
Macular edema is the common cause of visual degeneration in people with
diabetes which can significantly increases a person's chance of developing
retinopathy as the diabetes progresses.\cite{klein1984wisconsin} Proliferative
diabetic retinopathy is caused by blockages in blood vessels which can can
starve the retina of oxygen. The retina responds to this by attempting to
grow blood vessels of its own, however they tend to be abnormal and hinder
fluid secretion and leak blood into the vitreous humour. Symptoms of
prolific retinopathy do not usually occur until damage has already been
done and in that case a sufferer would experience seeing "floaters",
"shadows" and loss of vision.

\todo{discuss industrial accidents. retinal detachment}

