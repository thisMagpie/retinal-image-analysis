% Author: Magdalen Berns

\chapter{The Anatomy and Physics of Eye}

\label{anatomy} % For referencing

\lhead{\emph{The Anatomy of The Eye}}

\section{The Cornea}

The cornea is a transparent layer around 0.6mm thick, which curves over the
iris of the eye.\cite{yaylali1997corneal,thoft1983x,patel1994refractive}
With a mean refractive index of around 1.4 (about the same as water),
the cornea allows plenty of light to pass through. \Eref{eq:refractive}
shows Snell's Law of refraction for a light wave passing through two
different isotropic materials which have refractive indices $n_1$ and $n_2$,
respectively. The angle $\theta_1$ is normal to the boundary between $n_1$
and $n_2$ and the angle $\theta_2$ is normal to the boundary between $n_2$
and $n_1$.

\begin{equation}
n_1\sin\theta_1=n_2\sin\theta_2
\label{eq:refractive}
\end{equation}


The outermost surface of the cornea is made of epithelial cells which
are continually lost and replaced.\cite{jester1999cellular,hassell2010molecular}
The reproduction of cells is facilitated in part by the tear ducts of
the eye which  serve to moisten the eyes and remove harmful bacteria.\cite{holly1977tear}

As people grow older, the radius of their eyes tends to decrease and
the cornea itself takes on a more spherical shape.\cite{guirao2000optical}

\section{Aqueus Humor}

\section{The Lens}
After light passes through the cornea to the lens of the eye.

\section{Vitreous Humor}

The lens of the eye sits inside a bag with muscles attached to it which
change the focal length of the image which the light waves transmits.

\section{The Retina}
The retina is where light is changed from light waves into electrical signals
which we call brainwaves.

\section{Cones and Rods}

Cones and rods have photoreceptors which convert particular frequencies of
light waves into electrical signals so that the brain can interpret what
the eye has seen Whilst cones are not particularly sensitive to light,
they do aid to visual acuity because they grant us colour vision.\cite{}

A healthy/normal eye will pick up the full spectrum of colours.
\ref{fig:wavelengths} shows normalised absorbency against wavelength for red,
green and blue cones of the eye.

If all the eye cones are not working then this can cause blindness to occur\cite{}

\section{The Optic Nerve}
