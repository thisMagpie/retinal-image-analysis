% Author: Magdalen Berns

 \chapter{The Anatomy of The Eye}

\label{anatomy} % For referencing

\lhead{\emph{The Anatomy of The Eye}}

\section{The Cornea}

The cornea is transparent allowing light to pass through it.
The outermost surface of the cornea is made of epithelial cells.
\cite{hassell2010molecular}
\cite{jester1999cellular,hassell2010molecular} As people grow older, the radius
of the eye cornea tends to decrease. The cornea itself takes on a more
spherical shape.\cite{guirao2000optical}

\section{Aqueus Humor}

\section{The Lens}
After light passes through the cornea to the lens of the eye.

\section{Vitreous Humor}

The lens of the eye sits inside a bag with muscles attached to it which change
the focal length of the image which the light waves transmits.

\section{The Retina}
The retina is where light is changed from light waves into electrical signals
which we call brainwaves.

\section{Cones and Rods}

The function of cones and rods is to convert particular frequencies of light
waves
into electrical signals so that the brain can interpret what the eye has seen.
Whilst cones are not particularly sensitive to light, they do aid to visual
acuity.
\cite{}

A healthy/normal eye will pick up the full spectrum of colours.
\ref{fig:wavelengths} shows normalised absorbance against wavelength for red,
green and blue cones of the eye.

If all the eye cones are not working then this can cause
blindness to occur\cite{}

\section{The Optic Nerve}
