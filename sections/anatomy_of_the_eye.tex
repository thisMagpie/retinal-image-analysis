% Author: Magdalen Berns

\chapter{The Anatomy and Physiology of The Eye}

\label{anatomy}

The cornea is a transparent layer around 0.6mm thick, which curves over
the iris of the eye over its anterior chamber.
\cite{yaylali1997corneal,thoft1983x,patel1994refractive}
With a mean refractive index of around 1.4 (about the same as water),
the cornea allows plenty of light to pass through, but refracts as a
convergent lens due to its convex, shape. \Eref{eq:refractive} shows
Snell's Law of refraction for a light wave passing through two different
isotropic materials which have refractive indices $n_1$
and $n_2$, respectively.

\begin{figure}[htbp]
  \centering
    \includegraphics{figures/snells}
  \caption{Snell's law for diffraction at an interface where $n_1$ \textless $n_2$}
  \label{fig:snell}
\end{figure}

The angle $\theta_1$ is normal to the boundary between $n_1$ and $n_2$
and the angle $\theta_2$ is normal to the boundary between $n_2$ and $n_1$
as shown in \fref{fig:snell}

\begin{equation}
n_1\sin\theta_1=n_2\sin\theta_2
\label{eq:refractive}
\end{equation}

The radius of the human cornea tends to decrease with age and the cornea
itself, takes on a more spherical shape.\cite{guirao2000optical} Its
outermost surface is made of epithelial cells which are continuously lost
and replaced.\cite{jester1999cellular,hassell2010molecular} The reproduction
of cells is facilitated in part by tear ducts, which  serve to moisten the
eyes and remove harmful bacteria.\cite{holly1977tear}

There is a small circular opening in the iris, the pupil has an aperture
which dilates to allow an appropriate amount of refracted light to pass
through the lens. Light is refracted once again through the lens converges
towards a focal point on the back of the eye. The lens makers expression
in \eref{eq:lens_makers} is an expression described by the diagram in
\fref{convergent_lens} which shows light passing through a convex lens
 for calculating the focal point, $f$ of distances $S_1$ and $S_2$ either
  side of a given convex lens.\cite{greivenkamp2004field}

\begin{equation}
\frac{1}{S_1} + \frac{1}{S_2} = \frac{1}{f}
\label{eq:lens_makers}
\end{equation}

\begin{figure}[htbp]
  \centering
    \includegraphics{figures/convergent_lens2}
  \caption{light passing through a convex lens for calculating the focal
  point, $f$ of distances $S_1$ and $S_2$ either side of a given convex lens.}
  \label{fig:convergent_lens}
\end{figure}

A ciliary body of tissue made up of fiber and muscle which accommodates the
lens and secretes a fluid, known as Aqueous Humour into a canal that flows
around the circumference of the eye called the Scleral Venous Sinus.
\cite{bill1970effects,dvorak1934schlemm}

When it becomes necessary to focus on objects at short distances, the
ciliary body moves outwards and suspensory ligaments attached to the
posterior chamber and the posterior lens to become taut, causing the
lens to become somewhat thicker in response to applied contractile
forces.\cite{atchison1995accommodation}

When objects are far away the light is more parallel to
the lens' principle axis, less refraction is necessary, so
the muscles relax and suspensory ligaments attached to the
posterior chamber and the posterior lens so that it is
no longer being accommodated.\cite{}\fref{fig:cilary_processes}
shows some key ciliary processes. 

As humans age, ciliary muscles weaken and this impairs the ciliary muscles'
ability to accommodate the lens and the result of this is visual impairment.
\cite{fisher1985ciliary}

\begin{figure}[htbp]
  \centering
    \includegraphics{figures/cilary_processes}
  \caption{Labelled sketch of ciliary processes}
  \label{fig:cilary_processes}
\end{figure}

The retina is where light is changed from light waves into electrical
signals which we call brainwaves.\todo{write stuff}

\section{The Retina, Fovea, Cones and Rods}

Cones and rods have photoreceptors which convert particular frequencies of
light waves into electrical signals so that the brain can interpret what
the eye has seen Whilst cones are not particularly sensitive to light,
they do aid to visual acuity because they grant us colour vision.\cite{}

A healthy and normal eye will pick up the full spectrum of colours.
 \ref{fig:wavelengths} shows normalised absorbency against wavelength
 for red, green and blue cones for an average human eye.

\begin{figure}[htbp]
  \centering
    \includegraphics{figures/wavelengths}
  \caption{Normalised absorbency vs. wavelength, for an average eye.}
  \label{fig:wavelengths}
\end{figure}

If all the eye cones are not working then this causes blindness to occur\cite{}

\section{The Optic Nerve}
