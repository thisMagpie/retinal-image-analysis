\chapter{Introduction}

\label{intro}
\lhead{\emph{Introduction}}

There are many diseases that manifest themselves  in the  retina, thus
for many years scientists have made producing a detailed image of the
retina their life's work.  Although the primary  purpose of this technology
is for diagnostic and monitoring purposes, ophthalmologists are using
these devices to study the retina's structure to gain insight into the
manifestations of these diseases.  There are many physical principles
behind the technology used to image the retina; including understanding
the complex optics of the eye, utilising the optics of the eye to obtain an
image of the retina and the use of physically advanced systems to gain
insight into the retinal structure.

This review focuses on three main technologies in retinal 
imaging research: fundus cameras, \gls{cslo} and \Gls{oct}.
To introduce these technologies a detailed introduction 
to the eye and its dysfunction, pathology, and injury is provided 
along with a comprehensive overview of the early origins of 
ophthalmoscopy in the 19th century.  To highlight the shortcomings 
of the current state of these technologies, there is a section discussing 
the future research following the discussion on fundus imaging and 
optical coherence tomography.  This discussion covers the 
most recent advancements in imaging technology utilising adaptive 
optics and two-photon microscopy.  The purpose of this review is to 
introduce a complex optical system and provide an extensive 
investigation into the technology that underpins retinal imaging.
